\documentclass{article}
\usepackage{amsmath}
\usepackage{amssymb} 
\title{LA-HW1}
\author{Zahra Rafati}
\date{\today}

\begin{document}
\maketitle

\section*{Question 1}
\begin{enumerate}
    \item[(a)] \(\mathbf{z} + \mathbf{u} \stackrel{1}{=} \mathbf{u} + \mathbf{z} 
                \stackrel{3}{=} \mathbf{u}\).
    
    \item[(b)] Consider there exists a \(\mathbf{w}\) such that \(\mathbf{u} + \mathbf{w} 
                = \mathbf{u}\) for every \(\mathbf{u} \in \mathbf{V}\).\\
                \(\mathbf{u} + \mathbf{w} = \mathbf{u} \xrightarrow{\mathbf{u} = \mathbf{z}} 
                \mathbf{z} + \mathbf{w} = \mathbf{z} \stackrel{1}{=} \mathbf{w} + 
                \mathbf{z} \stackrel{3}{=} \mathbf{w}\).\\
                Therefore, \(\mathbf{z} = \mathbf{w}\).

    \item[(c)] \(\mathbf{z'} = \mathbf{w} \xrightarrow{+ \mathbf{z}} 
                \mathbf{z} + \mathbf{z'} = \mathbf{z} + \mathbf{w}\).\\
                \(\mathbf{z} + \mathbf{z'} \stackrel{4}{=} \mathbf{z}\).\\
                \(\mathbf{z} + \mathbf{w} \stackrel{a}{=} \mathbf{w}\).\\
                Thus, \(\mathbf{z} = \mathbf{w}\).

    \item[(d)] \(\mathbf{u} + \mathbf{w} = \mathbf{z} \stackrel{4}{=} \mathbf{u} + \mathbf{u'} 
                \xrightarrow{+ \mathbf{u}^{-1}} \mathbf{u'} + \mathbf{u} + \mathbf{w} 
                = \mathbf{u'} + \mathbf{u} + \mathbf{u'}\).\\
                Using (4, 1), \(\mathbf{z} + \mathbf{w} = \mathbf{z} + \mathbf{u'} 
                \xrightarrow{a} \mathbf{z} = \mathbf{u'}\).

    \item[(e)] \(\mathbf{u} + 0\mathbf{u} \stackrel{5}{=} 1\mathbf{u} + 0\mathbf{u} 
                \stackrel{8}{=} (1+0)\mathbf{u} = 1\mathbf{u} \stackrel{5}{=}
                \mathbf{u} \xrightarrow{\text{3b}} 0\mathbf{u} = \mathbf{z}\).

    \item[(f)] \(a\mathbf{z} \stackrel{e}{=} a(0\mathbf{u}) \stackrel{6}{=} 
                (a0)\mathbf{u} = 0\mathbf{u} \stackrel{e}{=} \mathbf{z}\).

    \item[(g)] \(0\mathbf{u} = (1-1)\mathbf{u} = 1\mathbf{u} + (-1)\mathbf{u} 
                = \mathbf{u} + (-1)\mathbf{u}\).\\
                \(\mathbf{z} = \mathbf{u} + \mathbf{u}^{-1}\).\\
                \(0\mathbf{u} = \mathbf{z} \xrightarrow{} 
                \mathbf{u} + \mathbf{u}^{-1} = \mathbf{u} + (-1)\mathbf{u} 
                \xrightarrow{d} \mathbf{u}^{-1} = (-1)\mathbf{u}\).
\end{enumerate}

\section*{Question 2}
Let
\[
\mathbf{A} \mathbf{x} = \mathbf{z}, \quad \text{where } \mathbf{A} = 
\begin{bmatrix}
a_1 & a_2 & \cdots & a_n
\end{bmatrix}, \quad \mathbf{x} = 
\begin{bmatrix}
x_1 \\ \vdots \\ x_n
\end{bmatrix}.
\]

\[
\mathbf{A} \mathbf{x} = x_1 \mathbf{a}_1 + x_2 \mathbf{a}_2 + \cdots + x_n \mathbf{a}_n.
\]

If \(\mathbf{a}_k = \mathbf{a}_k + 0\), then
\[
\mathbf{a}_k = \mathbf{a}_k + x_1 \mathbf{a}_1 + \cdots + x_n \mathbf{a}_n \implies x_1 \mathbf{a}_1 + \cdots + (x_k + 1)\mathbf{a}_k + \cdots + x_n \mathbf{a}_n = \mathbf{A} \mathbf{x'}.
\]

The problem states that for every \(\mathbf{x}\), the equation \(\mathbf{A} \mathbf{x} = 0\) holds. Therefore, we constructed a new \(\mathbf{x}\):
\[
\mathbf{A} \mathbf{x'} = 0 \implies \mathbf{a}_k = 0 \implies \mathbf{A} = \mathbf{0}_{m \times n}.
\]

\section*{Question 3}
Let
\[
\mathbf{x} = m_1 \mathbf{x}_1 + \cdots + m_n \mathbf{x}_n.
\]

Then,
\[
\mathbf{A} \mathbf{x} = m_1 \mathbf{A} \mathbf{x}_1 + \cdots + m_n \mathbf{A} \mathbf{x}_n \implies \mathbf{A} \mathbf{x} = 0.
\]

In Question 2, we proved that since \(\mathbf{x}\) can be any vector, if \(\mathbf{A} \mathbf{x} = 0\), then \(\mathbf{A} = 0\). Therefore:
\[
\mathbf{A} = \mathbf{0}_{m \times n}.
\]

\section*{Question 4}
\[
\mathbf{A} \mathbf{x} = \mathbf{x} \implies \mathbf{A} \mathbf{x} - \mathbf{x} = 0 \implies (\mathbf{A} - 1)\mathbf{x} = 0.
\]

\[
\mathbf{A} \mathbf{x} - \mathbf{I}_n \mathbf{x} = 0 \implies (\mathbf{A} - \mathbf{I}_n)\mathbf{x} = 0.
\]

Based on Question 2, if the above equation holds for every \(\mathbf{x}\), then \(\mathbf{A} - \mathbf{I}_n = 0\).

\[
\mathbf{A} = \mathbf{I}_n.
\]

\section*{Question 5}
Let
\[
\mathbf{A} = 
\begin{bmatrix}
1 & 0 & \cdots & 0 \\
0 & 0 & \cdots & 0 \\
\vdots & \vdots & \ddots & \vdots \\
0 & 0 & \cdots & 0
\end{bmatrix}, \quad
\mathbf{x} = 
\begin{bmatrix}
1 \\
0 \\
\vdots \\

0
\end{bmatrix}.
\]

Then,
\[
\mathbf{A} \mathbf{x} = 
\begin{bmatrix}
1 \\
0 \\
\vdots \\
0
\end{bmatrix}
= \mathbf{x}.
\]

\section*{Question 6}
Given:
\[
x_1 \mathbf{a}_1 + \cdots + x_n \mathbf{a}_n = 0 \implies \mathbf{x} = 
\begin{bmatrix}
x_1 \\
x_2 \\
\vdots \\
x_n
\end{bmatrix} = \mathbf{0}.
\]

Substituting:
\[
x_1 \mathbf{a}'_1 + \cdots + x_n \mathbf{a}_n = 0 \implies x_1(\mathbf{a}_1 + \beta \mathbf{a}_2) + \cdots + x_n \mathbf{a}_n = 0.
\]

\[
\implies x_1 \mathbf{a}_1 + (x_2 + \beta x_1)\mathbf{a}_2 + \cdots + x_n \mathbf{a}_n = 0.
\]

Since:
\[
x_1 = 0 \implies \beta x_1 = 0.
\]

And:
\[
x_2 = 0.
\]

Thus:
\[
x_2 + \beta x_1 = 0.
\]

\section*{Question 7}
\begin{enumerate}
\item[(a)]
\[
\langle \mathbf{A}, \mathbf{B} \rangle = \sum_{i=1}^m \sum_{j=1}^n \mathbf{A}_{ij} \mathbf{B}_{ij}.
\]

\[
\text{trace}(\mathbf{S}) = \sum_{i} \mathbf{S}_{ii}.
\]

Let \( \mathbf{A} \in \mathbb{R}^{m \times n} \) and \( \mathbf{B} \in \mathbb{R}^{n \times t} \), then:
\[
\mathbf{C}_{m \times t} = \mathbf{A}_{m \times n} \mathbf{B}_{n \times t}, \quad c_{ij} = \sum_{k=1}^n \mathbf{a}_{ik} \mathbf{b}_{kj}.
\]

%I
\[
\text{(I)} \quad (\mathbf{A}^T \mathbf{B})_{ij} = \sum_{k=1}^m \mathbf{a}_{ki} \mathbf{b}_{kj} \implies \text{trace}(\mathbf{A}^T \mathbf{B}) = 
\]

\[
\sum_{i=1}^n \sum_{k=1}^m \mathbf{a}_{ki} \mathbf{b}_{ki} = \sum_{k=1}^m \sum_{i=1}^n \mathbf{a}_{ki} \mathbf{b}_{ki} = \langle \mathbf{A}, \mathbf{B} \rangle.
\]

%II
\[
\text{(II)} \quad (\mathbf{B}^T \mathbf{A})_{ij} = \sum_{k=1}^m \mathbf{b}_{ki} \mathbf{a}_{kj} \implies \text{trace}(\mathbf{B}^T \mathbf{A}) =
\]

\[
\sum_{i=1}^n \sum_{k=1}^m \mathbf{b}_{ki} \mathbf{a}_{ki} = \sum_{k=1}^m \sum_{i=1}^n \mathbf{a}_{ki} \mathbf{b}_{ki} = \langle \mathbf{A}, \mathbf{B} \rangle.
\]

%III
\[
\text{(III)} \quad (\mathbf{A} \mathbf{B}^T)_{ij} = \sum_{k=1}^n \mathbf{a}_{ik} \mathbf{b}_{jk} \implies \text{trace}(\mathbf{A} \mathbf{B}^T) =
\]

\[
\sum_{i=1}^m \sum_{k=1}^n \mathbf{a}_{ik} \mathbf{b}_{ik} = \langle \mathbf{A}, \mathbf{B} \rangle.
\]

\item[(b)]
%IV
\[
\text{(IV)} \quad \langle \mathbf{A} \mathbf{B}, \mathbf{C} \rangle \overset{(I)}{=} \text{trace}((\mathbf{A} \mathbf{B})^T \mathbf{C}) = \text{trace}(\mathbf{B}^T \mathbf{A}^T \mathbf{C})
\]

\[
= \text{trace}(\mathbf{B}^T (\mathbf{A}^T \mathbf{C})) \overset{(I)}{=} \langle \mathbf{B}, \mathbf{A}^T \mathbf{C} \rangle.
\]

%V
\[
\text{(V)} \quad \langle \mathbf{A} \mathbf{B}, \mathbf{C} \rangle \overset{\text{(III)}}{=} \text{trace}(\mathbf{A} \mathbf{B} \mathbf{C}^T) = \text{trace}(\mathbf{A} (\mathbf{C} \mathbf{B}^T)^T)
\]

\[
= \langle \mathbf{A}, \mathbf{C} \mathbf{B}^T \rangle.
\]
\end{enumerate}

\section*{Question 8}
Let
\[
\mathbf{A} = 
\begin{bmatrix}
a_{11} & a_{12} & \cdots & a_{1n} \\
0 & a_{22} & \cdots & a_{2n} \\
\vdots & \vdots & \ddots & \vdots \\
0 & 0 & \cdots & a_{nn}
\end{bmatrix},
\]
where \( \mathbf{A} \) is an upper triangular matrix. We assume \( a_{kk} = 0 \) for \( 1 \leq k \leq n \). Our goal is to prove that there exists a non-zero vector \( \mathbf{x} \) such that \( \mathbf{A} \mathbf{x} = 0 \).

Define:
\[
\mathbf{A} = 
\begin{bmatrix}
\mathbf{A}_1 & \mathbf{A}_2 \\
0 & \mathbf{A}_3
\end{bmatrix}.
\]

Where \( \mathbf{A}_{3_{11}} = 0 \). 

\[
\mathbf{A} \mathbf{x} = 0 \implies 
\begin{bmatrix}
\mathbf{A}_1 & \mathbf{A}_2 \\
0 & \mathbf{A}_3
\end{bmatrix}
\mathbf{x} = 0 \implies 
\mathbf{x} = 
\begin{bmatrix}
-\mathbf{A}_1^{-1} \mathbf{A}_2 \mathbf{z} \\
\mathbf{z}
\end{bmatrix}.
\]

\[
\mathbf{A} \mathbf{x} =
\begin{bmatrix}
\mathbf{A}_1 & \mathbf{A}_2 \\
0 & \mathbf{A}_3
\end{bmatrix}
\begin{bmatrix}
-\mathbf{A}_1^{-1} \mathbf{A}_2 \mathbf{z} \\
\mathbf{z}
\end{bmatrix}
=
\begin{bmatrix}
-\mathbf{A}_1 \mathbf{A}_1^{-1} \mathbf{A}_2 \mathbf{z} + \mathbf{A}_2 \mathbf{z} \\
\mathbf{A}_3 \mathbf{z}
\end{bmatrix}
=
\begin{bmatrix}
-\mathbf{A}_2 \mathbf{z} + \mathbf{A}_2 \mathbf{z} \\
\mathbf{A}_3 \mathbf{z}
\end{bmatrix}
=
\begin{bmatrix}
0 \\
\mathbf{A}_3 \mathbf{z}
\end{bmatrix}.
\]

We can find a \( \mathbf{z} \) where \( \mathbf{A}_3 \mathbf{z} = 0 \).

\section*{Question 9}
\[
a (\mathbf{u} + \mathbf{v}) + b (\mathbf{u} + \mathbf{w}) + c (\mathbf{v} + \mathbf{w}) = \mathbf{0}.
\]

Expanding, we get:
\[
(a + b) \mathbf{u} + (a + c) \mathbf{v} + (b + c) \mathbf{w} = \mathbf{0}.
\]

From this, we have:
\[
a + b = 0, \quad a + c = 0, \quad b + c = 0.
\]

Solving the equations:
\[
a + b = 0 \implies a = -b,
\]
\[
a + c = 0 \implies a = -c,
\]
\[
b + c = 0 \implies c = -b.
\]

Substituting \( c = -b \) into \( a = -c \), we get:
\[
a = b.
\]

Thus, we have:
\[
a = b = c = 0.
\]

\section*{Question 10}
1. Non-emptiness:  
   For \( \mathbf{W} \), each vector \( \mathbf{W} \) is a subspace of \( \mathbf{V} \).  
   \[
   \mathbf{0} \in \mathbf{W} \implies \forall \mathbf{v} \in \mathbf{W}, \; \mathbf{0} \in \mathcal{M}.
   \]

2. Closed under addition:  
   If \( \mathbf{u}, \mathbf{v} \in \mathcal{M} \), then \( \mathbf{u} + \mathbf{v} \in \mathbf{W} \).  
   \[
   \forall \mathbf{u}, \mathbf{v} \in \mathbf{W}, \; \mathbf{u} + \mathbf{v} \in \mathcal{M}.
   \]

3. Closed under scalar multiplication:  
   Let \( \mathbf{u} \in \mathcal{M} \) and \( c \in \mathbb{F} \) (where \( \mathbb{F} \) is the field over which \( \mathbf{V} \) is defined).  
   \[
   \forall \mathbf{u} \in \mathbf{W}, \; c \mathbf{u} \in \mathbf{W}.
   \]

Since \( \mathbf{W} \) is closed under scalar multiplication:
\[
\forall \mathbf{u} \in \mathbf{W}, \; c \mathbf{u} \in \mathcal{M} \implies c \mathbf{u} \in \mathcal{M}.
\]


\end{document}
