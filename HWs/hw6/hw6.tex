\documentclass{article}
\usepackage{amsmath}
\usepackage{amssymb}
\title{LA-HW6}
\author{Zahra Rafati}
\date{\today}
\begin{document}
\maketitle

\section*{Question 1}
Using the singular value decomposition (SVD) of a matrix $A$, we have:
\[
\det(A) = \det(U) \det(\Sigma) \det(V^T),
\]
where $U$ and $V$ are orthogonal matrices, and $\Sigma$ is the diagonal matrix of singular values.

Since $\det(V^T) = \det(V)$:
\[
\det(A) = \det(U) \det(\Sigma) \det(V).
\]

Now, consider $\det(\Sigma) > 0$ (as the singular values are positive). The sign of $\det(A)$ is determined by the product:
\[
\text{sign}(\det(A)) = \text{sign}(\det(U) \det(V)).
\]

Since $\det(U)$ and $\det(V)$ are either $+1$ or $-1$ (as $U$ and $V$ are orthogonal):
\[
\det(U) \det(V) = \pm 1.
\]

Thus:
\[
\text{sign}(\det(U) \det(V)) = \text{sign}(\det(A)).
\]


\section*{Question 2}
The eigenvalue decomposition of a symmetric matrix $A$ is given by:
\[
A = V \Lambda V^T,
\]
where $\Lambda = \text{diag}(\lambda_1, \dots, \lambda_n)$ and all $\lambda_i > 0$ if $A$ is positive definite.

The singular value decomposition (SVD) of $A$ is:
\[
A = U \Sigma V^T.
\]

If $A$ is symmetric, then $U = V$. 

The matrix $\Sigma$ represents the singular values of $A$. For symmetric positive definite matrices, the singular values are the square roots of the eigenvalues of $A^T A$ or $A^2$.

If $A$ is positive definite, the eigenvalues of $A$ are positive. Thus:
\[
\text{Singular values } = \text{ Eigenvalues.}
\]

In conclusion:
\[
A = U \Sigma V^T = V \Lambda V^T, \quad \Sigma = \Lambda, \quad U = V.
\]


\section*{Question 3}
The eigenvalue decomposition of $A$ is given by:
\[
A = V \Lambda V^T,
\]
where $\Lambda = \text{diag}(\lambda_1, \dots, \lambda_n)$.

The singular value decomposition (SVD) adjusts $\Lambda$ to include absolute values of eigenvalues:
\[
\Sigma = \text{diag}(|\lambda_1|, \dots, |\lambda_n|).
\]

If $\lambda_i < 0$, then the corresponding eigenvector $\mathbf{v}_i$ is adjusted to $-\mathbf{v}_i$ in $V$ to ensure positivity in $\Sigma$.

The adjusted matrices $U$ and $V$ are used in the SVD representation:
\[
A = U \Sigma V^T.
\]


\section*{Question 4}
Let $B = P A Q$, where $P^T P = I$ and $Q^T Q = I$. Then:
\[
B^T B = (P A Q)^T (P A Q) = Q^T A^T P^T P A Q.
\]

Since $P^T P = I$, we have:
\[
B^T B = Q^T A^T A Q.
\]

Here, $Q$ is orthogonal, so the eigenvalues of $B^T B$ are equal to the eigenvalues of $A^T A$.

The singular values of $B$ are the square roots of the eigenvalues of $B^T B$. Therefore:
\[
\text{Singular values of } B = \text{Singular values of } A.
\]








\end{document}